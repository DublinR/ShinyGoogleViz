\documentclass{beamer}

\usepackage{amsmath}
\usepackage{amssymb}

\begin{document}
%-----------------------------------------------------------------------------------------------%
\begin{frame}
\frametitle{Dynamic UI}
\begin{itemize}
\item Shiny apps are often more than just a fixed set of controls that affect a fixed set of outputs. \item Inputs may need to be shown or hidden depending on the state of another input, or input controls may need to be created on-the-fly in response to user input.
\end{itemize}
\end{frame}
%-----------------------------------------------------------------------------------------------%
\begin{frame}
\frametitle{Dynamic UI}
Shiny currently has three different approaches you can use to make your interfaces more dynamic. From easiest to most difficult, they are:

\begin{itemize}
\item The \texttt{conditionalPanel} function, which is used in ui.R and wraps a set of UI elements that need to be dynamically shown/hidden
\item The \texttt{renderUI} function, which is used in server.R in conjunction with the htmlOutput function in \texttt{ui.R}, lets you generate calls to UI functions and make the results appear in a predetermined place in the UI

\item Use JavaScript to modify the webpage directly.
\end{itemize}
Let’s take a closer look at each approach.
\end{frame}

%-----------------------------------------------------------------------------------------------%
\begin{frame}
\frametitle{Dynamic UI}[fragile]
\textbf{Showing and Hiding Controls With conditionalPanel}\\
\texttt{conditionalPanel} creates a panel that shows and hides its contents depending on the value of a JavaScript expression. Even if you don’t know any JavaScript, simple comparison or equality operations are extremely easy to do, as they look a lot like \texttt{R} (and many other programming languages).

Here’s an example for adding an optional smoother to a \textit{\textbf{ggplot}}, and choosing its smoothing method:

\begin{verbatim}
# Partial example
checkboxInput("smooth", "Smooth"),
conditionalPanel(
  condition = "input.smooth == true",
  selectInput("smoothMethod", "Method",
              list("lm", "glm", "gam", "loess", "rlm"))
)
\end{verbatim}
\end{frame}
\end{document}


%-----------------------------------------------------------------------------------------------%
\begin{frame}
In this example, the select control for \texttt{smoothMethod} will appear only when the smooth checkbox is checked. Its condition is "\texttt{input.smooth == true}", which is a JavaScript expression that will be evaluated whenever any inputs/outputs change.

The condition can also use output values; they work in the same way (\texttt{output.foo} gives you the value of the output foo). If you have a situation where you wish you could use an \texttt{R} expression as your condition argument, you can create a reactive expression in \texttt{server.R} and assign it to a new output, then refer to that output in your condition expression. 
%For example:
\end{frame}

%-----------------------------------------------------------------------------------------------%
\begin{frame}[fragile]
\textbf{ui.R}
\begin{verbatim}
# Partial example
selectInput("dataset", "Dataset", c("diamonds", "rock", "pressure", "cars")),
conditionalPanel(
  condition = "output.nrows",
  checkboxInput("headonly", "Only use first 1000 rows"))
\end{verbatim}  
\textbf{server.R}
\begin{verbatim}
# Partial example
datasetInput <- reactive({
   switch(input$dataset,
          "rock" = rock,
          "pressure" = pressure,
          "cars" = cars)
})

output$nrows <- reactive({
  nrow(datasetInput())
})
\end{verbatim}
\end{frame}
%-----------------------------------------------------------------------------------------------%
\begin{frame}
However, since this technique requires server-side calculation (which could take a long time, depending on what other reactive expressions are executing) we recommend that you avoid using output in your conditions unless absolutely necessary.
\end{frame}
%-----------------------------------------------------------------------------------------------%
\begin{frame}
\frametitle{Creating Controls On the Fly With renderUI}
Note: This feature should be considered experimental. Let us know whether you find it useful.

Sometimes it’s just not enough to show and hide a fixed set of controls. Imagine prompting the user for a latitude/longitude, then allowing the user to select from a checklist of cities within a certain radius. In this case, you can use the renderUI expression to dynamically create controls based on the user’s input.
\end{frame}
%-----------------------------------------------------------------------------------------------%
\begin{frame}[fragile]
ui.R
\begin{verbatim}
# Partial example
numericInput("lat", "Latitude"),
numericInput("long", "Longitude"),
uiOutput("cityControls")
\end{verbatim}  
\textbf{server.R}
\begin{verbatim}
# Partial example
output$cityControls <- renderUI({
  cities <- getNearestCities(input$lat, input$long)
  checkboxGroupInput("cities", "Choose Cities", cities)
})
\end{verbatim}
\end{frame}
%-----------------------------------------------------------------------------------------------%
\begin{frame}
renderUI works just like renderPlot, renderText, and the other output rendering functions you’ve seen before, but it expects the expression it wraps to return an HTML tag (or a list of HTML tags, using tagList). These tags can include inputs and outputs.

In \texttt{ui.R}, use a uiOutput to tell Shiny where these controls should be rendered.
\end{frame}
%-----------------------------------------------------------------------------------------------%
\begin{frame}
\frametitle{Use JavaScript to Modify the Page}
Note: This feature should be considered experimental. Let us know whether you find it useful.

You can use JavaScript/jQuery to modify the page directly. General instructions for doing so are outside the scope of this tutorial, except to mention an important additional requirement. 
\end{frame}
%-----------------------------------------------------------------------------------------------%
\begin{frame}
Each time you add new inputs/outputs to the DOM, or remove existing inputs/outputs from the DOM, you need to tell Shiny. Our current recommendation is:
Before making changes to the DOM that may include adding or removing Shiny inputs or outputs, call Shiny.unbindAll().
After such changes, call \texttt{Shiny.bindAll()}.
If you are adding or removing many inputs/outputs at once, it’s fine to call Shiny.unbindAll() once at the beginning and \texttt{Shiny.bindAll()} at the end – it’s not necessary to put these calls around each individual addition or removal of inputs/outputs.
\end{frame}
%-----------------------------------------------------------------------------------------------%

\end{document}
