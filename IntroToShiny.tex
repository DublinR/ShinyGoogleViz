\documentclass[]{article}
\usepackage{amsmath}
\usepackage{framed}

%opening
\title{Shiny}
\author{Dublin \texttt{R}}

\begin{document}

\maketitle

\section{Introduction to Shiny}

\subsection{Introducing Shiny}
\textbf{Shiny} is a new package from \textit{RStudio} that makes it incredibly easy to build interactive web applications with \textit{R}.

% For an introduction and live examples, visit the Shiny homepage.

\subsection{Features of Shiny}
\begin{itemize}
\item Build useful web applications with only a few lines of code—no JavaScript required.
\item \textbf{Shiny} applications are automatically “live” in the same way that spreadsheets are live. Outputs change instantly as users modify inputs, without requiring a reload of the browser.
\item \textbf{Shiny} user interfaces can be built entirely using R, or can be written directly in HTML, CSS, and JavaScript for more flexibility.
\item Works in any R environment (Console R, Rgui for Windows or Mac, ESS, StatET, RStudio, etc.)
Attractive default UI theme based on Twitter Bootstrap.
\item A highly customizable slider widget with built-in support for animation.
Pre-built output widgets for displaying plots, tables, and printed output of \texttt{R} objects.
\item Fast bidirectional communication between the web browser and R using the websockets package.
\item Uses a reactive programming model that eliminates messy event handling code, so you can focus on the code that really matters.
\item Develop and redistribute your own \textbf{Shiny} widgets that other developers can easily drop into their own applications (coming soon!).
\end{itemize}
\subsection{Installation}
Shiny is available on CRAN, so you can install it in the usual way from your R console:
\begin{framed}
\begin{verbatim}
install.packages("shiny")
\end{verbatim}
\end{framed}

http://mathsci.ucd.ie/~parnell_a/WebApps/WebApps2.htm

%-------------------------------------------------------%
library(shiny)
runExample("01_hello")


Application Structure

The application consists of two main parts:

User Interface – ui.R
* Layouts
* Panels
* Inputs (selectInput, numericInput etc…)

Serverside – server.R
* function listeners

Architecture

The Shiny architecture is made up of a number of parts internally, namely a set of web/browser components including:

Twitter Bootstrap – a UI library that provides scaffolding in terms of layouts with HTML and CSS
Websockets – a way to do bi-directional communication with a browser/webserver.
Serverside Shiny makes use of Node.js for hosting (it doesn’t have to), but this to probably the most scalable approach.

Finally, it’s based on the concept of reactive programming which provides a number of parts:

Triggers/Inputs – Source (Shiny Reactive Value)
Sinks/Outputs – Endpoint (Shiny Observer)
The triggers are executed when dependencies (inputs) change resulting in a change to the output:

 Bash | 		 copy |	?	 
shinyServer(function(input, output) {
 output$plotOut <- renderPlot({
   hist(faithful$eruptions, breaks = as.numeric(input$nBreaks))
   if (input$individualObs)
     rug(faithful$eruptions)
 })
 
 output$tableOut <- renderTable({
   if (input$individualObs)
     faithful
   else
     NULL
 })
})
In some cases, changes could take a long time to run (accessing database, read from file). Shiny proposes conductors to encapsulate these operations. Shiny calls these reactive expressions

More info here – http://rstudio.github.io/shiny/tutorial/#reactivity-overview

Deployment

In terms of deployment you have a number of options:

1. You can deploy apps on hosted Shiny Server

2. Run from a Github paste (Gist) or from Github (this is an awesome feature)

 Bash | 		 copy |	?	 
shiny::runGist('3239667')
shiny::runGitHub('shiny_example', 'rstudio')
3. Deploy on Amazon EC2

4. Host yourself by installing Apache and NodeJS

Hope that gives you a good overview of Shiny, come along to the meetup if you want to learn more about it.
\end{document}
