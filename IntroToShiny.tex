\documentclass[]{article}
\usepackage{amsmath}
\usepackage{framed}

%opening
\title{Shiny}
\author{Dublin \texttt{R}}

\begin{document}

\maketitle

\section{Introduction to Shiny}

\subsection{Introducing Shiny}
\textbf{Shiny} is a new package from \textit{RStudio} that makes it incredibly easy to build interactive web applications with \textit{R}.

% For an introduction and live examples, visit the Shiny homepage.

\subsection{Features of Shiny}
\begin{itemize}
\item Build useful web applications with only a few lines of code—no JavaScript required.
\item \textbf{Shiny} applications are automatically “live” in the same way that spreadsheets are live. Outputs change instantly as users modify inputs, without requiring a reload of the browser.
\item \textbf{Shiny} user interfaces can be built entirely using R, or can be written directly in HTML, CSS, and JavaScript for more flexibility.
\item Works in any R environment (Console R, Rgui for Windows or Mac, ESS, StatET, RStudio, etc.)
Attractive default UI theme based on Twitter Bootstrap.
\item A highly customizable slider widget with built-in support for animation.
Pre-built output widgets for displaying plots, tables, and printed output of \texttt{R} objects.
\item Fast bidirectional communication between the web browser and R using the websockets package.
\item Uses a reactive programming model that eliminates messy event handling code, so you can focus on the code that really matters.
\item Develop and redistribute your own \textbf{Shiny} widgets that other developers can easily drop into their own applications (coming soon!).
\end{itemize}
\subsection{Installation}
Shiny is available on CRAN, so you can install it in the usual way from your R console:
\begin{framed}
\begin{verbatim}
install.packages("shiny")
\end{verbatim}
\end{framed}


\end{document}
